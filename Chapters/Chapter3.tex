% Chapter 1

\chapter{Adaptive task splitting} % Main chapter title

\label{Chapter3} % For referencing the chapter elsewhere, use \ref{Chapter3}

\lhead{Chapter 3. \emph{Adaptive task splitting}} % This is for the header on each page - perhaps a shortened title

%----------------------------------------------------------------------------------------
\section{An alternative to Rayon}
In the context of high performance computing, the abstraction provided by Rayon currently comes at a high cost of task creation overhead. This is because the heuristic for splitting is decoupled from availability of workers. The fundamental issue being that tasks are created irrespective of whether there are idle threads to steal them or not. This heuristic has been designed to reduce the idle time as much as possible, but that unfortunately doesn't help if the total overhead of excessive task creation dominates.

It is also important to note that Rayon splits the complete iterator in half. However, it is trivially true that splitting half of the \emph{remaining} iterator is better for load balancing. One can relate this to the sunk cost fallacy wherein the computation that has already taken place is now irrelevant to load balancing considerations from that point onwards.

As described in the previous section, Rayon splits tasks only till a specific depth is reached. This also implies that if the depth has been reached by all threads, but there is somehow an idle thread in the system, there would be no task splitting and hence that thread would remain idle forever.

In conclusion, the programmer would hence pay for an extremely high task creation overhead, while the load distribution would be more skewed towards the threads that started first.

This motivates the design and implementation of an Adaptive API developed in-house at INRIA.

\subsection{The adaptive task splitting heuristic}
In the adaptive API, the thread that starts the computation regularly listens on a channel for steal requests. Upon identifying one such request, it splits the remaining work into two halves and gives the latter half to the stealer. In case there are nested parallel iterators, the inner iterators will not create tasks until the outer ones have stopped splitting. This would ensure that the stolen tasks are coarse grained.

%\subsection{Collecting partial results}

\subsection{Contributions to the API}
As a part of this thesis, some changes were made to the API interface. Previously, the usage of the API was slightly more tedious with the constraint of having to use some types exported by this API. This meant that the API would not actually provide parallel iterators, but would require the programmer to implement a split function on a predefined container. This would allow the internal scheduler of the API to split the work and create tasks. This interface was completely changed (in the context of this thesis) to support Rayon parallel iterators. After these changes, it is sufficient to create a Rayon parallel iterator and call some adaptive methods on it (while the Adaptive API is in scope). This would bypass Rayon's scheduler and follow the adaptive task splitting.
