% Chapter 1

\chapter{The Rust programming language} % Main chapter title

\label{Chapter1} % For referencing the chapter elsewhere, use \ref{Chapter1}

\lhead{Chapter 1. \emph{About Rust}} % This is for the header on each page - perhaps a shortened title

%----------------------------------------------------------------------------------------

\section{The need for Rust}
While modern day languages provide higher amounts of abstraction and ease of programming, they either come at the cost of safety, or runtime performance, or both. Systems development is still dominated by C/C++ due to the high performance that they offer. However, it is extremely easy to create fundamental loopholes in huge systems that may lead to invalid memory access and crash the entire system, if the programmer is lucky. In the worst case, there may be data corruption due to an invalid access that is extremely hard to trace.

Rust addresses the dire need for a fast and safe systems development language. It goes above and beyond in this regard to offer zero cost abstractions in the form of abstract traits and types that allow seamless integration of new features with an existing system. Furthermore, it also allows for functional programming with its lazy iterators API. This generates highly optimized machine code with performance comparable to C.

Compilable code written in Rust can never lead to data races, dangling pointers, double frees or memory leaks. This comes from a simple sacrifice of the mutable state. By imposing the invariant that each memory location must have a single mutable reference to it (that can not overlap in time with an immutable reference), all above problems are prevented. Furthermore, it allows the language to offer automated memory management at minimal overhead as opposed to traditional methods of garbage collection.
%----------------------------------------------------------------------------------------

\section{Some quirks of programming in Rust}

The invariant of not having multiple mutable references to a memory object severely restricts syntactic expression of any given algorithm. The language hence suffers from a steep initial learning curve. The further sections shall elaborate some jargon that describes the memory model.

\subsection{Ownership}
The concept of ownership imposes the following rules[Cite the book here]:
\begin{itemize}
\item Each value in Rust has a variable that is its \emph{owner}.
\item This owner has mutable access to the value/memory object in question.
\item There can be only one owner at a given time (read: scope).
\item When the owner goes out of scope, the value is dropped.
\end{itemize}
Here it is important to note that all the analysis regarding ownership is carried out at compile time, and hence calls to drop objects are inserted by the compiler. This therefore provides memory safety and management at minimal runtime overhead.

\subsection{Movement and copy}
When variables are reassigned or passed around across functions, they adopt exactly one of two semantics, move and copy. Typically, any variable on the heap adopts the former while lightweight datatypes residing on the stack adopt the latter.

The move semantic changes the owner of the variable. This means that the scope of the variable is now the scope of it's new owner. The previous owner is an invalid reference to the variable, and the compiler would trigger compile-time errors for any attempt to use the previous owner.

The copy semantic on the other hand creates a deep copy of the data contained in the variable. We now have two different variables at separate locations in the memory, and having separate owners. The data that they contain, however, is the same.


\subsection{Borrows and lifetimes}
The former design requires unnecessary moves for situations (for example in case a function intends to read some data, it must be moved in and out). As an alternative, the language offers it's own notion of references. When a reference is created to any variable, it is called a borrow of that variable. Each reference has a lifetime that defines how long the reference is usable. This lifetime can never be more than that of the data referred to. Again, these checks are carried out at compile time. By a way of default, all references are immutable. However, by explicitly obtaining the reference with the \emph{'\&mut'} specifier, it is possible to get a mutable reference. As shown in Figure \ref{fig:borrow_checker}, the compiler throws a compiler time error if an immutable reference and a mutable reference are held on the same object.

\begin{figure}%[opt-rat]
	%\centering
    \includegraphics[scale=0.25]{Pictures/borrow_checker.jpg}
		%\rule{35em}{0.5pt}
    \caption[Optimization]{The "borrow checker" in the compiler guarding against lost updates.}
	\label{fig:borrow_checker}
\end{figure}
Furthermore, the compiler imposes a check to ensure that the lifetimes of two mutable references, (or that of one mutable reference with an immutable reference) do(es) not overlap.

\subsection{Generic types and traits}
The language offers powerful abstractions in the form of generic types and \emph{traits}. A \emph{trait} is essentially a set of functions (and possibly some types as well). Typically, one or more functions in the trait are not implemented. They have a particular signature to be followed as is. However, one or many functions can be defined and implemented in the trait. In such a case, after implementing just the abstract functions, the programmer has the rest of the functions of the trait at his disposal without having to implement them. The best example of this is the \textbf{Iterator} trait. After providing the implementation of the \textbf{next()} method, the entire functional API is available for use on the given iterator. This feature is analogous to interfaces in Java, for example. Traits can be extended from each other in order to create an object oriented hierarchy.

This dovetails with the notion of a generic type. Rust offers the possibility of using generic types in trait or function definition. A generic type is constrained by certain trait implementations. This essentially means that the signature is valid for any and all types that implement the given trait(s). This is extremely useful since any future types being introduced in the system would automatically have various functionalities supported with a few trait implementations. Consider the following example:

\begin{verbatim}
fn largest<T: PartialOrd + Copy>(list: &[T]) -> T {
    let mut largest = list[0];
    for &item in list.iter() {
        if item > largest {
            largest = item;
        }
    }
    largest
}
\end{verbatim}

This function uses a generic type \textbf{T} and describes how to find the "\emph{largest}" element out of a list of elements of type \textbf{T} (it will borrow this list). The type \textbf{T} is generic in that it can be anything that implements ParitalOrd and Copy traits.
