% Chapter 1

\chapter{A visualisation library for Rayon} % Main chapter title

\label{Chapter4} % For referencing the chapter elsewhere, use \ref{Chapter1}

\lhead{Chapter 4. \emph{Rayon Logs}} % This is for the header on each page - perhaps a shortened title

%----------------------------------------------------------------------------------------
\section{Beyond benchmarking and profiling}
Comparing performance is traditionally achieved by benchmarking various alternative implementations and then possibly profiling some of them to take a closer look at how performance can be improved. However, in the case of parallel iterators, this becomes more complicated. Benchmarking only gives a relative estimate of the performance, while profiling parallel iterators is not currently supported. This motivates the in-house development of a visualisation library (called Rayon Logs) that allows to create SVGs of parallel iterators wherein task creation and stealing can be seen. This has been indispensable to compare the Adaptive API with Rayon time and again. The following sections describe various features of this framework, many of which have been improved upon within the scope of this thesis.
%----------------------------------------------------------------------------------------
\section{Visualisation of a parallel algorithm}
The Rayon Logs library illustrates each task that was created during the execution of the algorithm, as a box. The lenth of a box in the drawings is indicative of the amount of time spent in that task. These boxes are arranged in a hierarchial tree-like representation of the various tasks (a task placed below another one implies that the latter must end before the former starts). It also adds the relevant edges in case the outputs of multiple tasks have been merged together.

Each thread in the system is represented by a color, and the boxes are filled with the color corresponding to the thread that ran the task. Furthermore, the color of the box also depicts the speed at which the given task was executed. The speed of a task is defined as the size of input divided by the time taken. This speed is then normalised across all tasks of a given type, and a shade of the color is assigned to each box. A darker shade indicates a slow task. This allows for an instantaneous visual comparison of the speed of execution, which might be the limiting constraint in the speedup.

Each box furthermore, has an annotation that displays the:
\begin{itemize}
\item Exact time spent in the given task.
\item Size of the input of the given task.
\item The ID of the thread that executed the given task.
\item The speed with which the task was executed. This is defined as the work divided by the time.
\end{itemize}

This display concludes with a group of boxes, wherein the length of each box indicates the amount of idle time spent by the given thread. This SVG as a whole is animated so that one can understand which tasks ran in parallel with each other. All this can be seen in Figure \ref{fig:adaptive_log}.
\begin{figure}%[opt-rat]
	%\centering
    \includegraphics[scale=0.4]{Pictures/adaptive_log.png}
		%\rule{35em}{0.5pt}
    \caption[Optimization]{An image generated from the Adaptive API solving the infix problem}
	\label{fig:adaptive_log}
\end{figure}

%----------------------------------------------------------------------------------------
\section{Visual comparison of algorithms}
This library supports an experimentation API in which we can add several algorithms and compare their runs.
The API exports a thread pool to which these algorithms can be added.
Each algorithm is then run for a specified number of times.
This hence generates a common histogram with a distribution of various run times for each algorithm.
Furthermore, it also displays some statistics such as mean and median run times.
It also offers the possibility of tagging particular parts of the algorithm for which the time spent will be measured separately.

Such a log terminates with visualisations of median runs and best runs of all the algorithms.
%----------------------------------------------------------------------------------------
\clearpage
\section{Changelog}
During this thesis, the following improvements were made to the Rayon Logs library:

\begin{itemize}
\item The speed normalisation for the experimentation API was changed from per algorithm normalisation to normalisation across all algorithms. This required some major changes to the internal data-flow of loging.
\item The experimentation API itself was modified to support any number of algorithms being compared. Earlier it could only compare two algorithms. This was achieved by reimplementing the thread pool (to which these algorithms were attached) with a builder pattern.
\item Added computation and display for idle times in the experimentation API. The idle time has been defined as the difference of the total run time and sum of the time taken by all logged tasks.
\item Added computation and display for median run times in the experimentation API. This would be useful since there is weak correlation between mean run time and mean idle time, however there is a stronger correlation between the total runtime and the idle time of the median run of each algorithm.
\item Integrated the experiments API with the existing Hwloc-RS library for thread binding. The aforementioned thread pool can hence be bound to specific cores of a NUMA machine with two possible binding policies. One may choose to use all cores of a given NUMA node first before using the cores in another NUMA node, or may bind threads to cores in a round robin fashion across available NUMA nodes.
\item Completely redesigned parallel iterator logging. Previously, the leaf level tasks were being displayed in the case of parallel iterators, with no hierarchial information. This did not permit illustration of nested parallel iterators. This overhaul of logging semantics now generates a hierarchial display of iterators being split and fused. It can also properly display the nested parallel iterators in this hierarchy.
\end{itemize}

%----------------------------------------------------------------------------------------
%\clearpage
\section{A complete autogenerated log}

%\begin{figure}%[opt-rat]
	%\centering
    \includepdf[scale=0.8, pages=-, offset=75 -75]{Figures/logs.pdf}
		%\rule{35em}{0.5pt}
    %\caption[Optimization]{The log generated by the Experimentation API}
	%\label{fig:complete_log}
%\end{figure}
